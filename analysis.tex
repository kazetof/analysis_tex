\documentclass[oneside,openany]{jbook}
\usepackage{amsmath}
\usepackage{amsthm}
\usepackage{amsfonts}

\theoremstyle{definition}
\newtheorem{axiom}{公理}
\newtheorem{theorem}{定理}
\newtheorem*{theorem*}{定理}
\newtheorem{definition}{定義}
\newtheorem*{definition*}{定義}
\renewcommand\proofname{\bf 証明}
\newtheorem{lemma}[theorem]{補題}
\newtheorem{example}{例}[section]


\title{解析入門}
\author{Kazeto Fukasawa}
\date{\today}

\begin{document}
\maketitle
\tableofcontents


\chapter{実数}
\section{実数の定義}
\subsection{体}

\begin{definition}{(体)}

集合$\mathbb{K}$の任意の二つの元$a, b \in \mathbb{K}$に対して,和$a+b$と積$ab$の二つの演算が定義され,以下の性質を満たす時,$\mathbb{K}$は体(Field)であると呼ぶ.


\begin{description}
\item{(F1)} $a + b = b + a$. (和の交換律)
\item{(F2)} 任意の$a, b, c \in \mathbb{K}$について,$(a+b)+c = a+(b+c)$.(和の結合律)
\item{(F3)} ある$0 \in \mathbb{K}$が存在して,任意の$a \in \mathbb{K}$について,$a+0=a$.(和の単位元の存在)
\item{(F4)} 任意の$a \in \mathbb{K}$について,ある$-a \in \mathbb{K}$が存在して,$a + (-a)=0$となる.(和の逆元の存在)
\item{(F5)} $ab = ba$.(積の交換律)
\item{(F6)} 任意の$a,b,c \in \mathbb{K}$について,$(ab)c=a(bc)$.(積の結合律)
\item{(F7)} ある$1 \in \mathbb{K}$が存在して,任意の$a \in \mathbb{K}$について$a1 = a$.(積の単位元の存在)
\item{(F8)} $0$でない任意の$a \in \mathbb{K}$について,ある$a^{-1} \in \mathbb{K}$が存在して,$aa^{-1}=1$.(積の逆元の存在)
\item{(F9)} 任意の$a,b,c \in \mathbb{K}$について,$a(b+c)=ab+ac$.(和と積の分配律)
\item{(F10)} $1 \neq 0$.($0$以外の元の存在)
\end{description}

\end{definition}

\begin{theorem}{(和の単位元の一意性)}\label{theorem:sum_identity_elem_unique}
(F3)の和の単位元$0$は一意的である.
\end{theorem}

\begin{proof}
ある$0^{\prime} \in \mathbb{K}$が存在して,任意の$a \in \mathbb{K}$について,$a+0^{\prime}=a$を満たし,かつ$0 \neq 0^{\prime}$となると仮定する.
このとき,$0 \in \mathbb{K}$より$0^{\prime}$の性質から$0 + 0^{\prime} = 0$が成り立つ.一方で,$0^{\prime}  \in \mathbb{K}$より同様に$0$の性質から,$0^{\prime} + 0 = 0^{\prime}$である.また,F1より$0+0^{\prime}=0^{\prime}+0$であるため,$0=0+0^{\prime}=0^{\prime}+0=0^{\prime}$となり,$0=0^{\prime}$で矛盾である.
従って,$\exists 0^{\prime} \in \mathbb{K} \forall a \in \mathbb{K}(a + 0^{\prime} = a) \land 0\neq0^{\prime}$を否定した$\lnot (\exists 0^{\prime} \in \mathbb{K} \forall a \in \mathbb{K}(a + 0^{\prime} = a) ) \lor 0 = 0^{\prime}$が成り立つ.(F3)より和の単位元は常に存在するため,$0 = 0^{\prime}$が常に成り立つ.
\end{proof}

\begin{theorem}
$\forall a \in \mathbb{K} (0a = 0)$.
\end{theorem}

\begin{proof}
(F3)で$a=0$とすると$0+0=0$であるため,$0a = (0 + 0)a$となる.ここで,(F9)の分配律を用いると,$0a = (0+0)a=0a+0a$である.$0a =0a+0a$は,(F3)より$0a$が$0a$和の単位元となっていることを意味する.定理\ref{theorem:sum_identity_elem_unique}より,和の単位元は一意的であるため,$0a = 0$が成り立つ.
\end{proof}


\subsection{全順序集合}

\section{連続性の公理}
\begin{definition}{(デデキントの切断)}

$\mathbb{R}$を全順序集合である体とする.このとき,$\mathbb{R}$の部分集合の組$( A, B)$について以下が成り立つとき,$( A, B)$をデデキントの切断と呼ぶ.

\begin{enumerate}
\item $A \neq \emptyset \land B \neq \emptyset$.
\item $A \cap B = \emptyset$.
\item $A \cup B = \mathbb{R}$.
\item $\forall a \in A \forall b \in B (a < b)$.
\end{enumerate}

\end{definition}

\begin{axiom}{(デデキントの公理)}\label{axiom:d_axiom}

デデキントの切断に対して,$A$の最大元と$B$の最小元の存在に関して,以下の4ケースが考えられる.

\begin{enumerate}
\item $A$の最大元が存在する $\land$ $B$の最小元が存在しない.
\item $A$の最大元が存在しない $\land$ $B$の最小元が存在する.
\item $A$の最大元が存在する $\land$ $B$の最小元が存在する.
\item $A$の最大元が存在しない $\land$ $B$の最小元が存在しない.
\end{enumerate}

このとき,1,2のみに限る.
\end{axiom}

3.は$\mathbb{R}$が全順序集合である体であることから常に成り立たない.この公理は,4があり得ないことを仮定する.

\begin{theorem}{($\mathbb{R}$の稠密性)}

$\mathbb{R}$を全順序集合である体とする.このとき,$\mathbb{R}$は稠密である.
\end{theorem}

\begin{proof}
任意の$a,b \in \mathbb{R}$($a < b$)について,$c=\frac{a+b}{2}$ととる.$\mathbb{R}$が体であることより,$c \in \mathbb{R}$であり,全順序集合であることからこの$c$にも順序関係が定義される.このときの順序は,$a < b \iff a + a < a + b \iff a = \frac{a + a}{2} < \frac{a + b}{2} = c$より$a < c$.同様に,$a < b \iff a + b < b + b \iff c = \frac{a + b}{2} < \frac{b + b}{2} = b$より$c < b$である.したがって任意の$a,b \in \mathbb{R}$($a < b$)について$a < c < b$となる$c$が存在するため,$\mathbb{R}$は稠密である.$\square$
\end{proof}

\begin{theorem}

公理\ref{axiom:d_axiom}デデキントの切断の

\begin{enumerate}
\setcounter{enumi}{2}
\item $A$の最大元が存在する $\land$ $B$の最小元が存在する.
\end{enumerate}
は,常に成り立たない.
\end{theorem}

\begin{proof}
3.が成り立つと仮定し,$A$の最大元を$a$,$B$の最小元を$b$とすると,$a=b$はありえない.$a \in A \land a \in B$となり切断の定義2. $A \cap B = \emptyset$.に反するからである.$a \neq b$のとき切断の定義4. より$a < b$である.$\mathbb{R}$の稠密性より$a < c < b$となる$c \in \mathbb{R}$が存在する.この$c$は$a < c$より$c \notin A$であり$c < b$より$c \notin B$である.これは切断の定義3. $A \cup B = \mathbb{R}$.に矛盾する.$\square$
\end{proof}

\begin{axiom}{(有界性公理)}
$\mathbb{R}$を全順序集合である体とする.$\mathbb{R}$の任意の部分集合$A \subset \mathbb{R} (A \neq \emptyset)$について,$A$が上に有界(下に有界)ならば,上限$s = \sup A \in \mathbb{R}$(下限$s = \inf A$)が存在する.
\end{axiom}


\subsection{デデキントの公理と有界性公理の同値性}
ここで,デデキントの公理と有界性公理が同値であることを示す.

\subsubsection{デデキントの公理 $\Rightarrow$ 有界性公理}
有界性公理の,上に有界な空でない任意の部分集合$A \subset \mathbb{R}$について,

\begin{align}
\begin{split}
U = \{ u \in  \mathbb{R} | \forall a \in A ( a \leq u) \},\\
L = U^{C} = \{ l \in \mathbb{R} | \exists a \in A ( a > l) \}.
\end{split}
\end{align}

と定義する.$U$はAの上界であり,$L$はその補集合である.

\begin{enumerate}
\item $(L, U)$が切断である.
\item $L$の最大元または$U$の最小元が$s = \sup A$となる.
\item $L$の最大元は存在しない.
\item $U$の最小元が$s = \sup A$となる.
\end{enumerate}

という順に示す.

まずデデキントの切断であることを確認する.

1. $L \neq \emptyset \land U \neq \emptyset$.

$A$は有界であるため,ある$b \in \mathbb{R}$が存在し,$\forall a \in A (a \leq b)$となる.このとき,$b \in U$であるため$U \neq \emptyset$である.また,$A \neq \emptyset$の仮定より,ある$a \in A$がとれる.$a-1 \in \mathbb{R}$を考えると,$a-1 < a \in A$であり,これは$a-1 \in L$を意味する.従って$L \neq \emptyset$となる.

2. $L \cap U = \emptyset$.
$L$は$U$の補集合として定義しているため,$L \cap U = \emptyset$である.実際に,$L \cap U \neq \emptyset$とすると,$c \in L \cap U$がとれて,$c \in U$より$\forall a \in A (a \leq c)$である.一方で,$c \in L$より$\exists a \in A (a > c)$となり任意の$a$について$a \leq c$に矛盾する.従って$L \cap U = \emptyset$が示された.

3. $L \cup U = \mathbb{R}$.
$L=U^{C}$であったため,$U^{C} \cup U = \mathbb{R}$を示す.まず,$U, U^{C} \subset \mathbb{R}$より$U \cup U^{C} \subset \mathbb{R}$が成り立つ.実際に,任意の$u \in U \cup U^{C}$について,$u \in U$であるとき$U \subset \mathbb{R}$より$u \in \mathbb{R}$であり,$u \in U^{C}$のときも同様に$U^{C} \subset \mathbb{R}$より$u \in \mathbb{R}$であるため,$U \cup U^{C} \subset \mathbb{R}$が成り立つ.一方で,任意に$a \in \mathbb{R}$をとると,$a \in U$または$a \notin U$のいずれかが成り立つ.従って任意の$a \in \mathbb{R}$について,$$a \in \mathbb{R} \Rightarrow a \in \{ a | a \in U \lor a \notin U \} = \{ a | a \in U \} \cup \{ a | a \notin U \} = U \cup U^{C}$$より,$\mathbb{R} \subset U^{C} \cup U$.以上より$U^{C} \cup U = \mathbb{R}$が示された.

4. $\forall a \in L \forall b \in U (a < b)$.
任意に$a \in L$をとると,$a < c$となる$c \in A$が存在する.また,$U$の定義より任意の$c^{\prime} \in A$に対して,任意の$b \in U$は$c^{\prime} \leq b$が成り立つ.任意の$c^{\prime}$について成り立つため,$c$についても成り立ち,$a < c \leq b$となる.以上より,$\forall a \in L \forall b \in U (a < b)$が示された.

1.2.3.4より,$(L, U)$は切断である.デデキントの公理より,$L$の最大元または$U$の最小元が存在する.

$L$の最大元は存在しない
$L$の最大元が存在すると仮定し,それを$s$とおく.このとき$s \in L$より$s < a \in A$となる$a$が存在する.$\mathbb{R}$の稠密性より,$s < b < a$となる$b \in \mathbb{R}$が存在し,$s$が$L$の最大元であることから,$b \in U$が成り立つ.しかし,$b < a$が任意の$a \in A$について$a \leq u$が成り立つという$U$の定義に反する.従って$L$の最大元は存在しない.

$U$の最小元は$A$の上限
$L$の最小元が存在しないため,デデキントの公理より$U$の最小元が存在する.$U$は$A$の上界の集合として定義されていたため,その最小元は$A$の上限である.従って$A$の上限が存在する.$\square$

\subsubsection{有界性公理 $\Rightarrow$ デデキントの公理}

まず,任意の切断$(L, U)$の$L$は上に有界である.もしも$L$が上に有界でないとすると,任意の$a \in \mathbb{R}$について$a < l$となる$l \in L$が存在し,$U = \emptyset$となるが,これは切断の定義2.に反する.

この$L$に対して有界性公理より上限が存在する.$s = \sup L$とおくと,$s \in L$と$s \in U$の場合が考えられる.$s \in L$であるとき,$s$は$L$の最大元であり,$s \in U$であるとき,$s$は$U$の最小元であることを示す.

$s \in L$のとき,$s$は$L$の上界の要素であることから,$\forall l \in L(l \leq s)$が成り立ち,仮定より$s \in L$であるため,$s$は$L$の最大元である.

$s \in U$のとき,$\forall u \in U(s \leq u)$を示せば良い.任意の$\epsilon > 0$について,$s - \epsilon$は$s$が$L$の上限であることから,$L$に属する.つまり,$\forall x \in \mathbb{R}(x < s \Rightarrow x \in L)$が成り立ち,対偶をとると$\forall x \in \mathbb{R}(x \in U \Rightarrow x \geq s)$となる.従って$s$は$U$の最小元である.$\square$



\begin{theorem}
\label{theorem:monotonic_inc_conv}
$\mathbb{R}上の,$上に有界な単調増加数列$\{ a_{n} \}$は収束する.
\end{theorem}

\begin{proof}
$\{ a_{n} \}$が上に有界な単調増加数列であるとき,それを集合とみなした$A = \{ a_{n} | n \in \mathbb{N}\}$は,上に有界な$\mathbb{R}$の部分集合である.そのため,有界性公理より上限$s = \sup(A)$が存在する. \\
$\epsilon > 0$を任意にとると,$s$が上限であることより,$s-\epsilon$は$A$の上界ではない. $s-\epsilon$が$A$の上界ではないとき,$s-\epsilon < a_{n_{0}} \leq s$ となるような$a_{n_{0}} \in A$が存在する. \\
このとき,$\{ a_{n} \}$が単調増加数列であることを用いると,任意の$n_{0} \leq n$となる$n$について,
$$s-\epsilon < a_{n_{0}} \leq a_{n}$$
が成り立つ.また,$s$は上限であり任意の$a_{n} \in A$について$a_{n} \leq s$を満たすため,
$$s-\epsilon < a_{n_{0}} \leq a_{n} \leq s$$
が成り立つ.$\epsilon > 0$より右辺に$\epsilon$を足すと等号ではなくなり,
$$s-\epsilon<a_{n}< s + \epsilon \iff  | a_{n} - s| < \epsilon$$
となる.\\
$\epsilon > 0$について$a_{n_{0}} \in A$が存在するとき,$\{ a_{n} \}$は$\mathbb{N} \to A$の全射とみなせるため,$a_{n_{0}}$に対応する$n_{0} \in \mathbb{N}$は存在する.\\
最終的に,任意の$\epsilon > 0$について$n_{0} \in \mathbb{N}$が存在し,$n_{0} \leq n \Rightarrow  | a_{n} - s| < \epsilon$を満たす.これは,$\{ a_{n} \}$が$s$に収束することを意味する.
\end{proof}

\begin{theorem}{(アルキメデスの定理)}
任意の二つの$a, b \in \mathbb{R}$について,$na > b$となる$n \in \mathbb{N}$が存在する.
\end{theorem}

\begin{proof}
この定理は,$b$が数列$(na)_{n \in \mathbb{N}}$の上界ではないことを主張している.$b$が数列$(na)_{n \in \mathbb{N}}$の上界であると仮定すると,$(na)_{n \in \mathbb{N}}$は単調増加数列であるため,定理\ref{theorem:monotonic_inc_conv}よりある値$\alpha \in \mathbb{R}$に収束する.このとき,収束の定義より,$\forall \epsilon > 0 \exists  n_{0} \in \mathbb{N} (  n \geq n_{0} \Rightarrow | na - \alpha | < \epsilon )$が成り立つため,$n \geq n_{0}$のとき,

\begin{align}
\begin{split}
\alpha - \epsilon < na,\\
na < \alpha + \epsilon, 
\end{split}
\end{align}

の両方が成り立つ.ここで$\epsilon = a$とすると,$n_{a} \in \mathbb{N}$が存在して,$n \geq n_{a}$のとき,

\begin{align}
\begin{split}
\alpha - a < na,\\
na < \alpha + a.
\end{split}
\end{align}

1つ目の不等式の両辺に$2a$を足すと,$\alpha + a < (n+2)a$となり,$n_{a} < n+2$であるため,$n+2$は二つ目の不等式を満たし,$(n+2)a < \alpha + a$となり,これは矛盾である.したがって$b$は数列$(na)_{n \in \mathbb{N}}$の上界ではない.
\end{proof}


\end{document}